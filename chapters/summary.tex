\section{Wyniki}
Wykonany projekt zrealizował cel jakim była konstrukcja klasyfikatora ziaren
rud miedzi, z~użyciem termowizji aktywnej oraz sieci neuronowych.
Zaproponowane sposoby zbierania danych, ekstrakcji cech oraz model sieci
okazały się wystarczające do osiągnięcia ponad dziewięćdziesięcio-procentowej
dokładności.
Jest to zadawalający wynik, świadczący o~dobrej jakości klasyfikacji.
W~toku projektu wyciągnięto liczne wnioski dotyczące alternatywnych
rozwiązań oraz możliwości rozwoju projektu.

Skonstruowany prototyp sieci dobrze pełni rolę klasyfikatora rud miedzi.
Zbudowane mechanizmy walidacji modelu mogą być pomocne przy jego
przyszłej rozbudowie oraz ocenie alternatywnych struktur sieci neuronowych.
Ze względu na bezkontaktowy i~niedestruktywny charakter pomiarów
termowizją aktywną, przedstawiono sposób klasyfikacji może znaleźć liczne
zastosowania.
Opracowane algorytmy mogą być podstawą do wykorzystania sieci neuronowych
do klasyfikacji ziaren rud miedzi w~środowisku przemysłowym.
Połączenie termowizji aktywnej oraz uczenia maszynowego może znaleźć
zastosowanie w~kontroli jakości półproduktów zawierających rudy miedzi.
Skonstruowane funkcje przetwarzania obrazów z~kamery, automatyczne kadrowanie
oraz odczytywanie temperatur ze zdjęć mogą być pomocne w~przyszłej pracy
z~wykorzystywanym sprzętem.
Zgromadzone materiały, techniki pomiarowe i~algorytmy mogą być pomocne
w~przyszłych pracach naukowych oraz dyplomowych bazujących na wykorzystaniu
sieci neuronowych w~klasyfikacji ziaren różnorakich materiałów.

\section{Wnioski}
Jak wspomniano w~sekcji \ref{sec:meas} proces budowy bazy danych był
czasochłonny oraz niepozbawiony niedokładności.
Zgromadzony zestaw nagrań okazał się wystarczający do spełnienia zadania
klasyfikacji rud miedzi, jednak możliwe są liczne usprawnienia stanowiska
pomiarowego.
Zgodnie z~opisem wykorzystywanej kamery przedstawionym w~sekcji
\ref{sec:camera}, istnieją liczne narzędzia pozwalające na automatyzację
procesu nagrywania materiałów wideo.
Wykorzystywany aparat został wyposażony w~bogaty zestaw sposobów komunikacji
z~komputerem oraz bibliotekę LabVIEW.
Istnieje zatem możliwość realizacji projektu rozbudowy stanowiska
laboratoryjnego termowizji w~celu automatyzacji procesu ogrzewania oraz
nagrywania próbek.
Pozwoliłoby to na znacznie szybsze utworzenie dużego zestawu danych oraz
poprawiłoby jakość zbieranych materiałów.
Budowa bardziej stabilnego i~zautomatyzowanego stanowiska pomiarowego
wyeliminowałaby ryzyko utraty ostrości obrazu pochodzącego z~kamery.
Dodatkowo rozbudowa stanowiska pozwoliłaby na realizację alternatywnej
metody pomiarów, polegającej na ogrzewaniu próbek do osiągnięcia zadanej
temperatury.

Zebrane materiały wideo zawierają wiele szczegółów oraz danych
charakteryzujących badane materiały.
W~podsekcji \ref{subsec:featureextr} rozpatrzono możliwe sposoby ekstrakcji
cech ze zgromadzonych nagrań.
Wybrana metoda śledzenia detali o~małej emisyjności okazała się odpowiednia
do wykorzystania jej w~klasyfikacji ziaren.
Przeprowadzona analiza świadczy, że uzyskany zestaw cech próbek dobrze
oddaje charakter fizykalnego procesu stygnięcia materiałów.
Opracowana technika ekstrakcji danych jest efektywna i~nie wymaga użycia
skomplikowanych modeli sieci neuronowych.
Rozpatrzono także alternatywne rozwiązania, które mogą również okazać się
skuteczne i~interesujące.
Po rozbudowie stanowiska oraz zebraniu większej ilości danych warta 
rozpatrzenia może okazać się możliwość wykorzystania nowoczesnych
konwolucyjnych sieci neuronowych.
Zebrane w~pracy materiały i~wnioski mogą posłużyć do realizacji projektów,
które rozwiną temat badania ziaren metodą termowizji.

Projekt spełnił swoje założenia, a~jego efektem jest działający klasyfikator
rud miedzi.
Wymienione wnioski mogą prowadzić do jego rozwoju oraz testów alternatywnych
rozwiązań.
Projekt może być podstawą do dalszych badań naukowych, jak i~prób wdrożenia
w~praktycznych zastosowaniach.

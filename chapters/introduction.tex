\section{Motywacja projektu}
Mielenie i~granularyzacja materiałów są istotnymi elementami wielu procesów
przemysłowych \cite{budzan_grains}.
Badanie stopnia rozdrobnienia, składu oraz kształtów ziaren są kluczowe dla
zapewnienia odpowiedniej jakości przetwarzania surowców.
W~ramach działalności Politechniki Śląskiej w~Gliwicach powstało wiele prac
naukowych oraz projektów związanych z~procesem mielenia materiałów.
Podczas badań opracowano, między innymi, systemy: sterowania procesem mielenia,
wizji komputerowej oraz klasyfikacji i~obróbki danych związanych z~ziarnami rud
miedzi.
W~czasie trwania projektów zebrano wiele próbek materiałów pochodzących
z~przemysłowych młynów surowców \cite{budzan_grains, krauze_milling}.

Politechnika Śląska jest także w~posiadaniu laboratorium termowizji, które
wyposażono w~kamery pracujące w~spektrum podczerwieni.
W~ramach projektu inżynierskiego zdecydowano się na realizację zadania detekcji
i~klasyfikacji ziaren rud miedzi z~użyciem termowizji oraz sieci neuronowych.
Ze względu na nieinwazyjny charakter pomiarów, techniki termowizyjne są
popularnym rozwiązaniem w~realizacjach przemysłowych.
Sieci neuronowe stanowią obecnie jedną z~najszybciej rozwijających się
i~obiecujących technik uczenia maszynowego.
Połączenie pomiarów termowizyjnych oraz sztucznych sieci neuronowych jest
interesującym tematem pracy.
Realizacja zadania klasyfikacji ziaren rud miedzi, z~użycie przedstawionych
technik, stanowi próbę alternatywnego rozwiązania problemów rozpatrywanych we
wcześniejszych pracach Politechniki Śląskiej.

\section{Cel pracy}
Celem pracy jest stworzenie systemu detekcji ziaren oraz klasyfikacji ich typu.
Punktem wyjścia projektu są próbki rud miedzi, które posiadają odgórnie
przypisany typ (klasę).
Budowa klasyfikatora polega na stworzeniu programu, który dla rudy miedzi
o~nieznanej klasie, będzie w~stanie określić jej typ.
W~rozpatrywanym zbiorze znajdują się cztery rodzaje rud miedzi, czyli należy
zbudować klasyfikator \emph{wieloklasowy}.

Zdecydowano się klasyfikować rudy miedzi według ich charakterystycznych wzorców
stygnięcia, do rejestrowania których służy kamera termowizyjna.
Na podstawie nagrań z~kamery należy utworzyć zbiór danych, który będzie podstawą
do treningu oraz oceny sieci neuronowej.
Ponieważ próbki przeznaczone do treningu sieci mają znane klasy, to zadanie
należy do grupy problemów \emph{nadzorowanego uczenia maszynowego}.
Oznacza to, że sieć będzie uczona poprzez podanie danych wejściowych oraz klasy
jaka jest oczekiwana na wyjściu.
Trening sieci będzie polegał na dopasowaniu jej do etykiet (klas) zbioru
uczącego.

Zadania uczenia maszynowego wykonuje się w~kilku etapach, które odpowiadają
rozdziałom pracy.
Kolejne kroki rozwiązania problemu klasyfikacji są następujące:
\begin{enumerate}
    \item analiza problemu oraz posiadanych materiałów,
    \item planowanie oraz przeprowadzenie eksperymentu uzyskania danych
          potrzebnych do nauki algorytmu klasyfikacji,
    \item analiza i~wizualizacja zebranych danych,
    \item przygotowanie danych i~ekstrakcja cech kluczowych dla
          algorytmów uczenia maszynowego,
    \item budowa modelu oraz jego nauka,
    \item dostrojenie modelu na podstawie systemu walidacji,
    \item prezentacja rozwiązania.
\end{enumerate}
Przedstawione kroki zadania klasyfikacji wymagają rozwiązania szeregu problemów
związanych z~rozpatrywanym przypadkiem.
Prace należy zacząć od zapoznania się z~dostępnym sprzętem termowizyjnym oraz
oceną jego użyteczności w~pomiarach ziaren rud miedzi.
Następnie należy zaplanować i~przeprowadzić eksperyment pomiarowy, którego
wynikiem będą nagrania ziaren.
Materiały wideo oraz zdjęcia zawierają wiele danych, należy je przeanalizować
oraz poddać obróbce.
Wynikiem eksploracji danych powinien być wektor liczb, które reprezentują
numerycznie cechy charakterystyczne rozpatrywanych klas rud miedzi.
Kolejnym krokiem jest konstrukcja sieci neuronowej w~postaci programu
komputerowego.
Sieć należy zaprojektować uwzględniając cechy zbudowanego zbioru danych.
Ocena sieci powinna przebiegać na podstawie wyników systemu walidacji.
Realizacja kolejnych etapów pracy prowadzi do celu jakim jest konstrukcja
działającego klasyfikatora rud miedzi.

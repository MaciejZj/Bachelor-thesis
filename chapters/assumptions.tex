\section{Analizowane pyły rud miedzi}

\section{Rodzaje termowizji i~idea wykorzystania pomiarów termowizyjnych}
\label{sec:thermovision}
Pomiary termowizyjne polegają na rejestracji promieniowania cieplnego obiektów
w~celu ustalenia ich temperatury.
Badania z~wykorzystaniem termowizji można podzielić na dwie grupy:
\emph{termowizję pasywną} oraz \emph{termowizję aktywną}.

Technika pasywna polega na rejestracji obrazów obiektów bez ingerencji w~ich
temperaturę w~czasie trwania pomiarów.
Można w~ten sposób obserwować przepływ ciepła w~urządzeniach technicznych,
procesach przemysłowych oraz biologicznych.
W~rozważanym w~pracy przypadku technika ta nie ma jednak zastosowania.
W~warunkach pokojowych badany materiał ma w~całej swojej objętości podobną
temperaturę co skutkuje obrazem równomiernego szumu w~rejestrowanym obrazie.

Bardziej zaawansowaną techniką są pomiary z~wykorzystaniem termowizji
aktywnej.
W tym trybie pomiarów badane obiekty nagrzewa się w~początkowej fazie pomiaru,
w~powtarzalnych warunkach, a~następnie rejestruje proces ich stygnięcia.
Podczas dostarczania ciepła do obiektu oraz jego oddawania na obrazach
termowizyjnych można zaobserwować strukturę badanego obiektu.
Przedmioty o~niejednorodnej i~złożonej budowie nagrzewają się oraz stygną
nierównomiernie co jest rejestrowane przez kamery termowizyjne.
W~ten sposób można dostrzec cechy obiektu niewidoczne gołym okiem, takie
jak zmiany jego gęstości, składu oraz uszkodzenia struktury.
Termowizję aktywną stosuje się szeroko w~badaniach naukowych oraz przemyśle.
Przykładowe aplikacje tej techniki to wykrywanie defektów w~produktach
przemysłowych oraz wykrywanie części konstrukcji podatnych na zużycie.
Termowizja aktywna jest techniką niedestruktywną oraz bezkontaktową,
co stanowi jej zalety w~analizie materiałowej.
Technika ta wymaga jednak etapowego procesu pomiarowego, potrzebne jest 
przygotowanie instalacji grzewczej, a~nagrzewanie i~stygnięcie materiału
może być czasochłonne.
Przebieg zmian temperatury najlepiej rejestrować na materiałach wideo,
aby zmaksymalizować ilość danych zgromadzonych w~trakcie eksperymentu.
% TODO: Może dać obrazek płytki z dziurami od dr Budzana

\section{Kamera termowizyjna FLIR A320}
Przedmiotem projektu jest analiza obrazów pochodzących z~przemysłowej
kamery termowizyjnej FLIR A320.
Firma FLIR zajmuje się produkcją wysokiej jakości kamer i~detektorów do celów
profesjonalnych i~przemysłowych.
Używana kamera łączący wysoką jakość pomiaru z~nowoczesnymi funkcjami
integracji z~oprogramowaniem komputerowym.
Urządzenie komunikuje się z~komputerem za pomocą kabla ethernetowego,
pozwalając na kontrolę z~poziomu oprogramowania oraz bibliotek
programistycznych.
Dodatkowo kamera posiada możliwość planowania automatycznych pomiarów,
alarmów, oferuje funkcje analityczne oraz wbudowany serwer internetowy.

\subsection{Opis sprzętowy wykorzystywanej kamery}
Kamera ma postać podłużnego korpusu, do którego swobodnie można podłączać
pasujące obiektywy.
Na rysunku \ref{fig:camera} przedstawiono zdjęcie wykorzystywanego urządzenia.
\begin{figure}[h]
    \centering
    \includegraphics[width=0.5\textwidth]{flir_a320}
    \caption{Kamera termowizyjna FLIR A320}
    \label{fig:camera}
\end{figure}
Egzemplarz kamery znajdujący się w~laboratorium termowizji Politechniki
Śląskiej został wyposażony w~obiektyw, pozwalający oglądać
ziarna rud miedzi w~powiększeniu.
% TODO: Datasheet bib ref or footnote
Kamera cechuje się następującymi parametrami:
\begin{itemize}
	\item typ detektora: niechłodzony mikrobolometr
	\item rozdzielczość: \num{320} na \num{240} pikseli,
	\item częstotliwość odświeżania: \num{9}\si{\hertz} do \num{30}\si{\hertz}
	\item szerokość otworu: \textit{f}\num{1,3},
	\item autofokus z wbudowanym silnikiem,
	\item zakres pomiarowy temperatur: 
		\begin{enumerate}
			\item od \num{-15}\si{\celsius} do \num{+50}\si{\celsius},
			\item od \num{0}\si{\celsius} do \num{350}\si{\celsius},
		\end{enumerate}
	\item dokładność: \num{\pm2}\si{\celsius} lub \num{\pm2}\% odczytu,
	\item zakres wykrywanego widma promieniowania: \num{7,5}\si{\micro\meter}
          do \num{13}\si{\micro\meter}.
\end{itemize}

Kamera wykrywa temperaturę przez detektor zwany \emph{bolometrem},
który mierzy energię niesioną przez fale elektromagnetyczne w~spektrum
podczerwieni.
Kiedy fala pada na detektor kamery, temperatura komórek matrycy rośnie
i~zwiększa się ich rezystancja elektryczna, co wpływa na natężenie prądu
przepływającego przez obwody w~czujniku.
Wartości natężenia prądu są mierzone, a~na ich podstawie określana jest
mierzona temperatura.

Zakres temperatur kamery jest odpowiedni do przeprowadzenia eksperymentów
z~pomiarami metodą termowizji aktywnej.
Próbki planuje się podgrzewać do temperatury maksymalnie około
\num{80}\si{\celsius}, wartość ta mieści się w~zakresie pracy urządzenia.
Dokładność kamery jest zadowalająca, próbne materiały nagraniowe
wskazały, że na zdjęciach widocznych jest wiele szczegółów i~detali
badanego materiału.
Przy klasyfikacji obrazów i~wzorców stygnięcia próbek jest to 
bardziej istotne niż liczbowa dokładność pomiarowa przyrządu.

W~porównaniu ze zwykłymi, współczesnymi aparatami rozdzielczość kamery
termowizyjnej może wydawać się bardzo mała.
Należy sobie jednak uzmysłowić, że w~standardowych aparatach piksele mają
rozkład Bayera, a~wartości składowych koloru są interpolowane.
W~kamerze termowizyjnej każdy piksel dokonuje pełnego pomiaru wartości
temperatury, dlatego bezpośrednie porównanie rozdzielczości używanego
przyrządu z~popularnymi aparatami może być mylące.
Oczywiście większa rozdzielczość kamery byłaby pożądana, jednak jej obecne
możliwości pozwalają na szczegółowe pomiary i~obserwacje wielu detali ziaren
rud miedzi.

\subsection{Obiektyw kamery}

Kamerę wyposażono w~obiektyw zbliżeniowy FLIR T19741, pozwalający obserwować
ziarna rud miedzy.
Jest to sprzęt zaprojektowany przez producentów używanej kamery i~dedykowany
pracy z~urządzeniami termowizyjnymi.
Przyrząd przedstawiono na rysunku \ref{fig:lens}.
\begin{figure}[h]
    \centering
    \includegraphics[width=0.5\textwidth]{flir_lens}
    \caption{Obiektyw FLIR T197415}
    \label{fig:lens}
\end{figure}
Wybrany obiektyw jest przygotowany z~myślą o~obserwacji drobnych detali
powierzchni w~dużym zbliżeniu.
Używany model ma następujące parametry:
\begin{itemize}
	\item ogniskowa: \num{18,2}\si{\milli\meter},
	% TODO: Powiększenie bez sensu
	\item powiększenie: \num{1x1},
	\item pole widzenia: \num{8}\si{\milli\meter} na \num{6}\si{\milli\meter},
	\item odległość od płaszczyzny ostrzenia: \num{20}\si{\milli\meter},
	\item głębia ostrości: \num{0,3}\si{\milli\meter},
	\item przysłona: bez regulacji, równa otworowi systemu montażu kamery,
	\item budowa: trzy soczewki asferyczne.
\end{itemize}

% TODO: Datasheet bib ref or footnote
Zgodnie ze specyfikacją producenta obiektyw ma powiększenie \num{1x1}, co może
wydawać się niedużą wartością.
Należy mieć jednak świadomość, że zwykłe obiektywy zmniejszają obraz
padający na matrycę, powszechnie przyjmuje się jako granicę makrofotografii
powiększenie \num{1x1}.
W~obserwacji ziaren i~detali powierzchni nie jest jednak istotne powiększenie,
ale to że używany obiektyw jest \emph{zbliżeniowy}.
Oznacza to, że ma on bardzo małą odległość ostrzenia, czyli można go przysunąć
blisko obserwowanej powierzchni.
Obiektyw zbliżeniowy pozwala na obserwację z~dystansu
\num{20}\si{\milli\meter}, typowe obiektywy ostrzą z~odległości parudziesięciu
centymetrów do ponad metra.
Dzięki temu obiektyw zbliżeniowy pozwala na obserwację bardzo drobnych
detali powierzchni.

\subsection{Oprogramowanie do obsługi kamery} \label{subsec:camsoft}
Jedną z~najważniejszych cech kamery jest łatwość jej integracji
z~oprogramowaniem komputerowym.
Kamerę można obsługiwać za pomocą programu FLIR Tools.
Pozwala on na podgląd obrazu oraz wykonywanie zdjęć termowizyjnych.
Producent dostarcza również bibliotekę LabVIEW pozwalającą na zaawansowaną
pracę z~kamerą.
Do obsługi stanowiska został napisany program używający tych bibliotek,
który pozwala na nagrywanie materiałów wideo przy pomocy kamery.
Nagrane materiały mają własnościowy format firmy FLIR, jednak można je
odtwarzać w~programie FLIR Tools.
Program pozwala także na eksport stopklatek z~nagrania, w~postaci plików JPEG.
Przy obsłudze narzędzia ważne jest ustawianie zakresu temperatur na obrazie.
Wybrany zakres decyduje w jaki sposób wartości temperatury są mapowane na
kolory w obrazie zawarte w tablicy LUT.
Program oferuje tablice w skali szarości, takie zostały użyte w~projekcie,
możliwy jest także wybór tablic w postaci kolorowych gradientów.
Wybór zakresu wpływa na wygląd wyświetlanego obrazu oraz eksportowanych
stopklatek.
Jego nieodpowiedni dobór może skutkować zbyt ciemnym, jasnym, lub mało
kontrastowym obrazem.
Aby zapewnić najlepsze wykorzystanie nagrań z~kamery oraz powtarzalny
charakter eksportu stopklatek korzystano z~opcji automatycznego doboru
zakresu temperatur, jaki jest wbudowany w~program FLIR Tools.

\section{Narzędzia programistyczne}

\subsection{Język programowania Python}
Założenia projektu wymagają użycia języka programowania pozwalającego
na zaawansowaną obróbkę obrazu oraz wydajne budowanie sieci neuronowych.
W~obu tych dziedzinach wiodącym językiem jest Python, posiadający bogaty
zestaw bibliotek.
Język ten oferuje dużą wygodę programowania, co jest istotne przy
prototypowaniu oraz pracy badawczej.
Jednocześnie Python jest zaopatrzony w~wydajną bibliotekę obliczeń
numerycznych NumPy.
Oferuje ona klasy macierzy numerycznych oraz bogaty zestaw operacji
matematycznych.
Macierze biblioteki NumPy są mniej elastyczne niż zwykłe listy języka Python,
jednak oferują dużo większą wydajność obliczeniową, między innymi dzięki
wsparciu obliczeń wektorowych na różnych procesorach.
Wiele innych bibliotek wykorzystuje pakiet NumPy, na przykład przy
przetwarzaniu obrazów, co pozwala na wysoką wydajność obliczeń.
Połączenie wygody programowania z~zadowalającą wydajnością bibliotek
numerycznych, sprawia że język Python jest dobrym wyborem do realizacji
założeń projektu.

\subsection{Biblioteka przetwarzania obrazu Scikit-image}
Język Python oferuje bogaty zestaw bibliotek przetwarzania obrazu.
Najbardziej popularną z~nich jest biblioteka OpenCV napisana w~języku C++.
Zdecydowano się jednak na wybór mniej znanego pakietu Scikit-image.
Jest to biblioteka nastawiona bardziej na obliczenia naukowe i~badawcze
oraz napisana bezpośrednio z~myślą o~języku Python.
Scikit-image, w~porównaniu z OpenCV, oferuje bardziej spójny interfejs
programisty oraz lepiej wykorzystuje charakterystykę języka Python.
Dodatkowo wybrana biblioteka jest częścią zestawu Scikit, w~skład
którego wchodzi pakiet Scikit-learn służący do uczenia maszynowego.
Może on być przydatny przy tworzeniu sieci neuronowej, a~spójność bibliotek
z~pakietu Scikit jest niewątpliwie zaletą.
Wybrana biblioteka charakteryzuje się również dobrą dokumentacją oraz
zestawem przydatnych przykładów.

\subsection{Interfejs sieci neuronowych Keras i~biblioteka TensorFlow}
Wybór bibliotek głębokiego uczenia w~języku Python jest bardzo szeroki.
Do najpopularniejszych pakietów należą: Keras, TensorFlow oraz PyTorch.
Zdecydowano się na wybór interfejsu biblioteki Keras.
Jest to pakiet nastawiony na elastyczność i~możliwość eksperymentowania,
oferuje wysokopoziomową, matematyczną warstwę abstrakcji opisu sieci
neuronowych.
Keras nie jest jednak pełnym rozwiązaniem, a~raczej abstrakcyjnym interfejsem.
Używanie go wymaga wyboru wewnętrznego silnika biblioteki.
Zdecydowano się na wybór domyślnej opcji i~użycie zaplecza pakietu TensorFlow.

\section{Analizowane ziarna rud miedzi} \label{sec:grains}
Zestaw analizowanych ziaren pochodzi z~projektu \textsc{sysmel} (\emph{System
sterowania procesem mielenia w~układzie z~młynem elektromagnetycznym})%
\footnote{Pełen tytuł projektu naukowego: \emph{Układ mielenia surowców
    mineralnych w~młynie elektromagnetycznym wraz z~systemem sterowania jego
    pracą, zapewniający wysoką efektywność technologiczną i~niską
    energochłonność w~zastosowaniach mikro i~makro-przemysłowych.}},
którego współtwórcami są naukowcy z~Politechniki Śląskiej.
Uzyskane próbki powstały w~procesie mielenia rud miedzi w~młynie
elektromagnetycznym.
Poszczególne klasy ziaren różnią się granulacją i~parametrami procesu ich
mielenia.
Spośród próbek, pochodzących z~projektu \textsc{sysmel}, wybrano cztery typy
ziaren, które będą przedmiotem zadania klasyfikacji.
Zdjęcia rozpatrywanych materiałów przedstawiono na rysunku \ref{fig:grains}.
\begin{figure}[htb]
    \hspace*{\fill}
    \begin{subfigure}[t]{0.35\textwidth}
        \centering
        \includegraphics[width=\textwidth]{E5R}
        \caption{Ziarna rudy miedzi klasy E5R}
    \end{subfigure}
    \hfill
    \begin{subfigure}[t]{0.35\textwidth}
        \centering
        \includegraphics[width=\textwidth]{E6R}
        \caption{Ziarna rudy miedzi klasy E6R}
    \end{subfigure}
    \hspace*{\fill}
    \vskip\baselineskip
    \hspace*{\fill}
    \begin{subfigure}[t]{0.35\textwidth}
        \centering
        \includegraphics[width=\textwidth]{E11R}
        \caption{Ziarna rudy miedzi klasy E11R}
    \end{subfigure}
    \hfill
    \begin{subfigure}[t]{0.35\textwidth}
        \centering
        \includegraphics[width=\textwidth]{E16R}
        \caption{Ziarna rudy miedzi klasy E16R}
    \end{subfigure}
    \hspace*{\fill}
    \caption{Próbki rozpatrywanych ziaren rud miedzi}
    \label{fig:grains}
\end{figure}

\section{Metody pomiarów termowizyjnych}
\label{sec:thermovision}
Pomiary termowizyjne (nazywane także \emph{termografią}) polegają na rejestracji 
promieniowania cieplnego obiektów w~celu ustalenia ich temperatury.
Badania z~wykorzystaniem termowizji można podzielić na dwie grupy:
\emph{termowizję pasywną} oraz \emph{termowizję aktywną}.

Technika pasywna polega na rejestracji obrazów obiektów bez ingerencji w~ich
temperaturę w~czasie trwania pomiarów.
Można w~ten sposób obserwować przepływ ciepła w~urządzeniach technicznych,
procesach przemysłowych oraz biologicznych.
W~rozważanym w~pracy przypadku technika ta nie ma jednak zastosowania.
W~warunkach pokojowych badany materiał ma w~całej swojej objętości podobną
temperaturę, co skutkuje obrazem równomiernego szumu w~rejestrowanym obrazie.

Bardziej zaawansowaną techniką są pomiary z~wykorzystaniem termowizji aktywnej.
W tym trybie pomiarów badane obiekty ogrzewa się w~powtarzalnych warunkach,
a~następnie rejestruje proces ich stygnięcia.
Podczas dostarczania ciepła do obiektu oraz jego oddawania na obrazach
termowizyjnych można zaobserwować strukturę badanego obiektu.
Przedmioty o~niejednorodnej i~złożonej budowie nagrzewają się oraz stygną
nierównomiernie, co jest rejestrowane przez kamery termowizyjne.
W~ten sposób można dostrzec cechy obiektu niewidoczne gołym okiem, takie jak 
zmiany jego gęstości, składu oraz uszkodzenia struktury.
Termowizję aktywną stosuje się szeroko w~badaniach naukowych oraz przemyśle.
Przykładowe aplikacje tej techniki to detekcja defektów w~produktach
przemysłowych oraz wykrywanie części konstrukcji podatnych na zużycie.
Termowizja aktywna ma charakter niedestruktywny oraz bezkontaktowy, co stanowi
jej zalety w~analizie materiałowej \cite{ciampa_thermography}.
Technika ta wymaga jednak etapowego procesu pomiarowego, potrzebne jest
przygotowanie instalacji grzewczej, a~nagrzewanie i~stygnięcie materiału może
być czasochłonne.
Przebieg zmian temperatury najlepiej rejestrować na materiałach wideo, aby
zmaksymalizować ilość danych zgromadzonych w~trakcie eksperymentu.

\section{Kamera termowizyjna \textsc{Flir~A320}}
\label{sec:camera}
Przedmiotem projektu jest analiza obrazów pochodzących z~przemysłowej kamery
termowizyjnej \textsc{Flir~A320}.
Firma \textsc{Flir} zajmuje się produkcją wysokiej jakości kamer i~detektorów do
celów profesjonalnych i~przemysłowych.
Używany aparat łączy wysoką jakość pomiaru z~nowoczesnymi funkcjami integracji
z~oprogramowaniem komputerowym.
Urządzenie komunikuje się z~komputerem za pomocą sieci internetowej, pozwalając
na kontrolę z~poziomu aplikacji oraz bibliotek programistycznych.
Kamera posiada także możliwość planowania automatycznych pomiarów i~alarmów,
oferuje również funkcje analityczne oraz wbudowany serwer internetowy
\cite{flir_a32x_manual}.

\subsection{Opis sprzętowy wykorzystywanej kamery}
Kamera ma postać podłużnego korpusu, do którego swobodnie można podłączać
obiektywy.
Na rysunku \ref{fig:camera}~przedstawiono zdjęcie wykorzystywanego urządzenia.
\begin{figure}[h]
    \centering
    \includegraphics[width=0.4\textwidth]{flir_a320}
    \caption{Kamera termowizyjna \textsc{Flir~A320} \cite{flir_camera_specs}}
    \label{fig:camera}
\end{figure}
Egzemplarz kamery znajdujący się w~laboratorium termowizji Politechniki Śląskiej 
został wyposażony w~obiektyw, pozwalający oglądać ziarna rud miedzi
w~powiększeniu.
Kamera cechuje się następującymi parametrami \cite{flir_camera_specs}:
\begin{itemize}
    \item typ detektora: niechłodzony mikrobolometr,
    \item rozdzielczość: 320 na 240 pikseli,
    \item częstotliwość odświeżania: \SI{9}{\hertz} do \SI{30}{\hertz},
    \item szerokość otworu: \textit{f}\num{1,3},
    \item autofokus z wbudowanym silnikiem,
    \item zakres pomiarowy temperatur:
          \begin{itemize}[label={$\diamond$}]
              \item od \SI{-15}{\celsius} do \SI{+50}{\celsius},
              \item od \SI{0}{\celsius} do \SI{350}{\celsius},
          \end{itemize}
    \item dokładność: \num{\pm2}\si{\celsius} lub \num{\pm2}\% odczytu,
    \item zakres wykrywanego widma promieniowania: \SI{7,5}{\micro\meter}
          do \SI{13}{\micro\meter}.
\end{itemize}

Kamera wykrywa temperaturę przez detektor zwany \emph{bolometrem}, który mierzy
energię niesioną przez fale elektromagnetyczne w~spektrum podczerwieni.
Kiedy fala pada na detektor aparatu, temperatura komórek matrycy rośnie
i~zwiększa się ich rezystancja elektryczna.
Wartości rezystancji są mierzone, a~na ich podstawie określana jest temperatura
\cite{vanhoof_infrared}.

Zakres temperatur kamery jest odpowiedni do przeprowadzenia eksperymentów
z~pomiarami ziaren rud miedzi metodą termowizji aktywnej.
Zaplanowano podgrzewanie próbek maksymalnie do temperatury około
\SI{80}{\celsius},~wartość ta mieści się w~zakresie pracy urządzenia.
Dokładność kamery jest zadowalająca, próbne materiały nagraniowe wskazały, że na
zdjęciach widocznych jest wiele szczegółów i~detali badanego materiału.
Przy klasyfikacji obrazów i~wzorców stygnięcia próbek jest to bardziej istotne
niż liczbowa dokładność pomiarowa przyrządu.

W~porównaniu ze zwykłymi, współczesnymi aparatami, rozdzielczość kamery
termowizyjnej może wydawać się mała.
Należy sobie jednak uzmysłowić, że w~standardowych aparatach piksele mają
rozkład Bayera, a~wartości składowych koloru są interpolowane.
W~kamerze termowizyjnej każda komórka matrycy dokonuje pełnego pomiaru wartości
temperatury.
Z~tego powodu bezpośrednie porównanie rozdzielczości używanego przyrządu
z~popularnymi aparatami może być mylące.
Oczywiście większa rozdzielczość kamery byłaby pożądana, jednak jej obecne
możliwości pozwalają na szczegółowe pomiary i~obserwację wielu detali ziaren
rud miedzi.

\subsection{Obiektyw kamery}
\label{subsec:lens}
Kamerę wyposażono w~obiektyw zbliżeniowy \textsc{Flir~T197415}, pozwalający
obserwować ziarna rud miedzi.
Jest to sprzęt zaprojektowany przez producentów używanej kamery i~przeznaczony
do pracy z~urządzeniami termowizyjnymi.
Przyrząd przedstawiono na rysunku \ref{fig:lens}.
\begin{figure}[h]
    \centering
    \includegraphics[width=0.5\textwidth]{flir_lens}
    \caption{Obiektyw \textsc{Flir~T197415} \cite{flir_lens_specs}}
    \label{fig:lens}
\end{figure}
Wybrany obiektyw jest przygotowany z~myślą o~obserwacji drobnych detali
powierzchni w~dużym zbliżeniu.
Używany model ma następujące parametry \cite{flir_lens_specs}:
\begin{itemize}
    \item ogniskowa: \SI{18,2}{\milli\meter},
    \item powiększenie: \num{1x1},
    \item pole widzenia: \SI{8}{\milli\meter} na \SI{6}{\milli\meter},
    \item odległość od płaszczyzny ostrzenia: \SI{20}{\milli\meter},
    \item głębia ostrości: \SI{0,3}{\milli\meter},
    \item przysłona: bez regulacji, równa otworowi systemu montażu kamery,
    \item budowa: trzy soczewki asferyczne.
\end{itemize}

Zgodnie ze specyfikacją producenta obiektyw ma powiększenie \num{1x1},~co może
wydawać się niedużą wartością.
Należy mieć jednak świadomość, że zwykłe obiektywy zmniejszają obraz padający na
matrycę.
Powszechnie jako granicę makrofotografii przyjmuje się powiększenie \num{1x1}.~%
W~obserwacji ziaren i~detali powierzchni nie jest jednak istotne powiększenie,
ale to że używany obiektyw jest \emph{zbliżeniowy}.
Oznacza to, że ma on bardzo małą odległość ostrzenia, czyli można go przysunąć
blisko obserwowanej powierzchni.
Obiektyw zbliżeniowy pozwala na obserwację z~dystansu
\SI{20}{\milli\meter},~typowe obiektywy ostrzą z~odległości parudziesięciu
centymetrów do ponad metra.
Dzięki temu obiektyw zbliżeniowy pozwala na obserwację bardzo drobnych detali
powierzchni.

\subsection{Oprogramowanie do obsługi kamery}
\label{subsec:camera_soft}
Jedną z~najważniejszych cech kamery jest łatwość integracji z~oprogramowaniem
komputerowym.
Kamerę można obsługiwać za pomocą programu \textsc{Flir} Tools%
\footnote{%
    Strona produktu programu \textsc{Flir} Tools:
    \url{https://www.flir.com/products/flir-tools/}
}.
Pozwala on na podgląd obrazu oraz wykonywanie zdjęć termowizyjnych.
Producent dostarcza również bibliotekę LabVIEW pozwalającą na zaawansowaną
pracę z~kamerą.
Do obsługi stanowiska został napisany program używający tych bibliotek, który
pozwala na nagrywanie materiałów wideo przy pomocy kamery.
Nagrania mają własnościowy format firmy \textsc{Flir}, można je odtwarzać
w~programie \textsc{Flir} Tools.
Program pozwala także na eksport stopklatek z~nagrania w~postaci plików
\textsc{jpeg}.
Przy obsłudze narzędzia ważne jest ustawianie zakresu temperatur na obrazie.
Wybrany zakres określa w~jaki sposób wartości temperatury są mapowane na kolory
na zdjęciu, zawarte w~tablicy \textsc{lut}.
Program oferuje tablice w~skali szarości, takie zostały użyte w~projekcie,
możliwy jest także wybór tablic w~postaci kolorowych gradientów.
Wybór zakresu wpływa na wygląd wyświetlanego obrazu oraz eksportowanych klatek.
Jego nieodpowiedni dobór może skutkować zbyt ciemnym, jasnym, lub mało
kontrastowym obrazem.
Aby zapewnić najlepsze wykorzystanie nagrań z~kamery oraz powtarzalny charakter
eksportu stopklatek korzystano z~opcji automatycznego doboru zakresu temperatur, 
jaki jest wbudowany w~program \textsc{Flir} Tools.

\section{Narzędzia programistyczne}

\subsection{Język programowania Python}
Założenia projektu wymagają użycia języka programowania pozwalającego na
zaawansowaną obróbkę obrazu oraz wydajne budowanie sieci neuronowych.
W~obu tych dziedzinach wiodącym językiem jest Python%
\footnote{%
    W~projekcie użyto języka Python 3,~który jest aktualnym standardem.
    Nie należy go mylić z~niekompatybilną i~przestarzałą wersją Python 2,~która
    traci wsparcie w~2020 roku.
}%
, posiadający bogaty zestaw bibliotek.
Język ten oferuje dużą wygodę programowania, co jest istotne przy pracy
badawczej.
Do wydajnych obliczeń numerycznych w~Pythonie służy biblioteka NumPy.
Oferuje ona klasy macierzy numerycznych oraz bogaty zestaw operacji
matematycznych.
Macierze biblioteki NumPy są mniej elastyczne od zwykłych list języka Python,
jednak oferują dużo większą wydajność obliczeniową, między innymi dzięki
wsparciu obliczeń wektorowych na odpowiednich procesorach.
Wiele innych bibliotek języka wykorzystuje pakiet NumPy, na przykład przy
przetwarzaniu obrazów, co pozwala na wysoką wydajność obliczeń.
Połączenie wygody programowania z~zadowalającą wydajnością bibliotek
numerycznych, sprawia że Python jest dobrym wyborem do realizacji założeń
projektu.

Język Python cechuje się także integracją z~notatnikami \emph{Jupyter}%
\footnote{Strona projektu Jupyter: \url{https://jupyter.org}.
}.
Notatniki stanowią interaktywny edytor kodu.
Ich zaletą jest możliwość programowania bezpośrednio wewnątrz dokumentu
formatowanego za pomocą języka \emph{Markdown}.
Taki sposób organizacji kodu ułatwia dokumentację i~prezentację wyników,
szczególnie w~przypadku prac o~charakterze badawczym i~naukowym.
Programy demonstrujące działanie algorytmów oraz przedstawiające analizę danych
opracowano w~formacie notatników Jupyter.

\subsection{Biblioteka przetwarzania obrazu Scikit-image}
Język Python oferuje bogaty zestaw bibliotek przetwarzania obrazu.
Najbardziej popularną z~nich jest biblioteka OpenCV napisana w~języku
C\texttt{++}.
Zdecydowano się jednak na wybór mniej znanego pakietu Scikit-image\footnote{%
\label{foot:version_req}Wersje używanych bibliotek są zawarte w~pliku
\texttt{requirements.txt} dołączonym do projektu.
Na jego podstawie można utworzyć wirtualne środowisko języka Python
z~odpowiednim zestawem bibliotek.
Więcej informacji o~strukturze projektu zawarto w~dodatku
\ref{ch:project_structure}.%
}.
Jest to biblioteka zorientowana na obliczenia naukowe i~badawcze oraz napisana
bezpośrednio z~myślą o~języku Python \cite{scikit-image}.
Scikit-image, w~porównaniu z OpenCV, oferuje bardziej spójny interfejs
programisty oraz lepiej wykorzystuje charakterystykę języka Python.
Dodatkowo wybrana biblioteka jest częścią zestawu Scikit, w~skład którego
wchodzi pakiet Scikit-learn służący do uczenia maszynowego.
Może on być przydatny przy tworzeniu sieci neuronowej, a~spójność bibliotek
z~pakietu Scikit jest niewątpliwie zaletą.
Wybrana biblioteka charakteryzuje się również dobrą dokumentacją
\cite{scikit_reference} oraz zestawem przykładów.

\subsection{Interfejs sieci neuronowych Keras}
\label{subsec:software_network}
Wybór bibliotek głębokiego uczenia w~języku Python jest bardzo szeroki.
Do najpopularniejszych pakietów należą: Keras, TensorFlow oraz PyTorch.
Zdecydowano się na wybór interfejsu biblioteki Keras%
\footnoteref{foot:version_req}.
Jest to pakiet nastawiony na elastyczność i~możliwość eksperymentowania
\cite{chollet_keras}.
Moduł Keras oferuje także rozbudowaną dokumentację oraz zestaw poradników
\cite{keras_docs}.
Keras nie jest jednak samodzielną biblioteką sieci neuronowych,
a~wyłącznie abstrakcyjnym interfejsem programistycznym.
Używanie go wymaga wyboru wewnętrznego silnika biblioteki.
Zdecydowano się na domyślną opcję użycia zaplecza pakietu TensorFlow.
Użyto wersji interfejsu Keras wchodzącej bezpośrednio w~skład modułu TensorFlow.
Taki sposób użycia bibliotek zapewnia ich najlepszą współpracę.


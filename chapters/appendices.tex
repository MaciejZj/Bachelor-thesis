\chapter{Porównanie konfiguracji i~parametrów modelu sieci}
\label{ch:nn_comparison_table}

\begin{table}[!h]
\centering
\begin{tabular}{c|c|c|c}
\toprule
Parametr                                              & Wartość   & Błąd & Dokładność \\
\midrule
\multirow{4}{*}{Funkcja aktywacji}                    & sigmoid   & 1,28 & 0,58       \\
                                                      & relu      & 1,71 & 0,83       \\
                                                      & elu       & 2,04 & 0,92       \\
                                                      & tanh      & 1,44 & 0,92       \\
\midrule
\multirow{2}{*}{Liczba warstw ukrytych}               & 1         & 1,71 & 0,67       \\
                                                      & 2         & 1,22 & 0,92       \\
\midrule                                                     
\multirow{3}{*}{\parbox{5cm}
{\centering Liczba neuronów w~warstwach ukrytych}}     & 128 i 64  & 1,47 & 0,83       \\
                                                      & 256 i 128 & 1,42 & 0,92       \\
                                                      & 512 i 126 & 0,97 & 0,92       \\
\midrule
\multirow{2}{*}{Algorytm uczenia}                     & sgd       & 0,96 & 0,58       \\
                                                      & adam      & 0,68 & 0,92        \\
\bottomrule
\end{tabular}
\end{table}

\chapter{Organizacja projektu}
\section{Struktura plików projektu}
Podczas budowania funkcji dbano o~ich zwięzłość, dobry podział realizowanych
zadań oraz ograniczenie długości bloków kodu.
Dzięki temu program zachowuje zasady dobrego programowania \cite{martin_code}.
Każdą funkcję opatrzono w~specjalny komentarz \emph{docstring} będący
standardem w~świecie języka Python.
Funkcje umieszczono w~odpowiednich plikach, które zawierają także
przykłady ich użycia i~obrazują działanie oraz porównują efekty.
Struktura plików w~projekcie jest następująca:
\begin{description}
    \item[img\_processing.py]
          funkcje przetwarzania obrazu i~wczytywania danych,
    \item[blob\_finder.py]
          funkcja wykrywania detali na zdjęciach termowizyjnych,
    \item[blob\_series\_tracker.py]
          funkcje śledzenia i~zliczania detali w~serii zdjęć,
    \item[neural\_network.py]
          funkcje dostępu do modelu sieci i~metod walidacji,
    \item[pliki demo.py]
          pliki demonstrujące działanie utworzonego oprogramowania,
    \item[thesis\_generator.py]
          funkcje generujące obrazy, wykresy oraz tabele znajdujące się
          w~treści tego dokumentu.
\end{description}

Projekt zaopatrzono w~plik \texttt{requirements.txt} zawierający listę
modułów bibliotecznych wymaganych przez projekt.
Stworzone programy najlepiej uruchamiać w~środowisku wirtualnym języka
Python \emph{virtualenv}\footnote{Dokumentacja narzędzia virtualenv:
\url{https://virtualenv.pypa.io/en/latest/}}.
Takie rozwiązanie pozwala na izolację projektu od innych modułów Pythona
zainstalowanych na komputerze oraz daje pewność zgodności wersji używanych
bibliotek.
Programując w~języku Python starano się utrzymać zgodność ze standardem
\emph{PEP8}\footnote{Strona standardu PEP8: \url{https://www.python.org/dev/
peps/pep-0008/}} zapewniającym wysoką jakość i~czytelność kodu.

\section{Narzędzia pomocnicze}
W~czasie programowania oraz pisania pracy korzystano z~systemu wersjonowania
Git.
Korzystanie z~systemu kontroli wersji zapewnia porządek w~rozwoju projektu
oraz chroni przed przypadkową utratą postępów.
W~celu zapewnienia jakości oprogramowania użyto programu Pylint\footnote{
Strona projektu Pylint: \url{https://www.pylint.org}} (\emph{lintera}
języka Python), analizującego kod w~celu wykrycia błędów.
Dokument opisujący projekt inżynierski napisano z~użyciem zestawu makr
LaTex oraz systemu bibliografii BibTex.


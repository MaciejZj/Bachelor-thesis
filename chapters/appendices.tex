\chapter{Porównanie konfiguracji i~parametrów modelu sieci}
\label{ch:nn_comparison_table}
\begin{table}[!h]
    \centering
    \begin{tabular}{c|c|c|c}
        \toprule
        Parametr & Wartość   & Błąd & Dokładność \\
        \midrule
        \multirow{4}{*}{Funkcja aktywacji}                 
            & sigmoid   & 1,28 & 0,58       \\
            & relu      & 1,71 & 0,83       \\
            & elu       & 2,04 & 0,92       \\
            & tanh      & 1,44 & 0,92       \\
        \midrule
        \multirow{2}{*}{Liczba warstw ukrytych}
            & 1         & 1,71 & 0,67       \\
            & 2         & 1,22 & 0,92       \\
        \midrule
        \multirow{3}{*}{\parbox{5cm}
            {\centering Liczba neuronów w~warstwach ukrytych}}
            & 128 i 64  & 1,47 & 0,83       \\
            & 256 i 128 & 1,42 & 0,92       \\
            & 512 i 256 & 0,97 & 0,92       \\
        \midrule
        \multirow{2}{*}{Algorytm uczenia}            
            & sgd       & 0,96 & 0,58       \\
            & adam      & 0,68 & 0,92       \\
        \bottomrule
    \end{tabular}
\end{table}

\chapter{Organizacja projektu}
\label{ch:project_structure}
\section{Struktura plików projektu}
Podczas pisania funkcji języka Python dbano o~ich zwięzłość, dobry podział
realizowanych zadań oraz ograniczenie długości bloków kodu.
Dzięki temu program zachowuje zasady dobrego programowania \cite{martin_code}.
Każdą funkcję opatrzono w~specjalny komentarz \emph{docstring} będący standardem
w~świecie języka Python.
Funkcje umieszczono w~odpowiednich plikach, które zawierają także przykłady ich
użycia i~obrazują działanie oraz porównują efekty.
Struktura plików w~projekcie jest następująca:
\begin{description}
    \item[img\_processing.py]
        funkcje przetwarzania obrazu i~wczytywania danych,
    \item[blob\_finder.py]
        funkcja wykrywania detali na zdjęciach termowizyjnych,
    \item[blob\_series\_tracker.py]
        funkcje śledzenia i~zliczania detali w~serii zdjęć,
    \item[neural\_network.py]
        funkcje dostępu do modelu sieci i~metod walidacji,
    \item[thesis\_generator.py]
        funkcje generujące obrazy, wykresy oraz tabele,
    \item[notatniki Jupyter]
        demonstracja utworzonego oprogramowania i~analiza danych,
    \item[docs] dokumentacja projektu.
\end{description}

Projekt zaopatrzono w~plik \texttt{requirements.txt} zawierający listę
wymaganych modułów bibliotecznych.
Stworzone programy najlepiej uruchamiać w~środowisku wirtualnym języka Python
\emph{virtualenv}%
\footnote{%
    Dokumentacja narzędzia virtualenv:
    \url{https://virtualenv.pypa.io/en/latest/}}.
Takie rozwiązanie pozwala na izolację projektu od innych modułów Pythona
zainstalowanych na komputerze oraz daje pewność zgodności wersji używanych
bibliotek.
Programując w~języku Python starano się utrzymać zgodność ze standardem
\emph{PEP8}%
\footnote{%
    Strona standardu PEP8: \url{https://www.python.org/dev/ peps/pep-0008/}}
zapewniającym wysoką jakość i~czytelność kodu.

\section{Dokumentacja projektu}
Za pomocą modułu Pydoc wygenerowano dokumentację projektu w~postaci plików
html.
Zawierają one listę utworzonych funkcji oraz ich opisy.
Pliki dokumentacji zawarto w~repozytorium projektu.

\section{Narzędzia pomocnicze}
W~czasie programowania oraz pisania pracy korzystano z~systemu wersjonowania
Git.
Używanie systemu kontroli wersji zapewnia porządek w~rozwoju projektu oraz
chroni przed przypadkową utratą postępów.
Aby zapewnić odpowiednią jakość oprogramowania użyto programu Pylint%
\footnote{%
    Strona projektu Pylint: \url{https://www.pylint.org}}
(\emph{lintera} języka Python), analizującego kod w~celu wykrycia błędów.
Dokument opisujący projekt inżynierski napisano z~użyciem zestawu makr LaTex
oraz systemu bibliografii BibTex.

\chapter{Spis wykorzystanych funkcji bibliotecznych}
\begin{description}
    \item[Biblioteka standardowa języka Python] moduły: \texttt{glob},
          \texttt{math} oraz \texttt{natsort}.
    \item[Keras] funkcje i~klasy z~przestrzeni nazw \texttt{keras}.
    \item[Matplotlib] funkcje z~przestrzeni nazw \texttt{mpatches} oraz
          \texttt{plt}.
    \item[Numpy] funkcje z~przestrzeni nazw \texttt{np}.
    \item[Pandas] klasa \texttt{DataFrame}.
    \item[Pillow] funkcje z~przestrzeni nazw \texttt{Image}.
    \item[Pytesseract] funkcje z przestrzeni nazw \texttt{pytesseract}.
    \item[Scikit-image] funkcje: \texttt{blob\_dog}, \texttt{blob\_doh},
          \texttt{blob\_log}, \texttt{crop}, \texttt{histogram},
          \texttt{img\_as\_ubyte}, \texttt{imread}, \texttt{imsave},
          \texttt{invert}, \texttt{rescale}, \texttt{rgb2gray} oraz
          \texttt{threshold\_otsu}.
    \item[Scikit-learn] funkcja \texttt{train\_test\_split} oraz klasa 
          \texttt{StratifiedKFold}.
    \item[TensorFlow] funkcje i~klasy z~przestrzeni nazw \texttt{tf}.
    \item[Tikzplotlib] funckje z przestrzeni nazw \texttt{tikzplotlib}.   
\end{description}

\section{Proces pomiarowy i~budowa zbioru danych} \label{sec:meas}
Zgodnie z~opisem technik termowizyjnych przedstawionym w~sekcji
\ref{sec:thermovision} zdecydowano się na przeprowadzenie pomiarów za pomocą
termowizji aktywnej.
Aby zrealizować pomiary uprzednio przygotowano stanowisko laboratoryjne.
Kamera termowizyjna została umieszczona na statywie, a~do ogrzewania próbek
zdecydowano się wykorzystać lampę halogenową.

W~procesie termowizji aktywnej istotna jest charakter procesu nagrzewania
materiału.
Przy przygotowaniu pomiarów należało zdecydować przy jakim warunku zakończyć
przekazywanie ciepła do próbki.
Rozważono dwie możliwości:
\begin{enumerate}[a)]
	\item ogrzewanie próbek do osiągnięcia ustalonej temperatury,
	\item \label{it:heatmethod} ogrzewanie próbek przez określony, stały czas.
\end{enumerate}
Zdecydowano się na metodę \ref{it:heatmethod}, ze względu na wygodę jej
realizacji.
Doprowadzenie każdej próbki do tej samej temperatury wymagałoby pomiarów
w~czasie nagrzewania, co jest bardziej wymagające do realizacji.
Zgodnie ze wstępnymi obserwacjami nagrzewanie materiału przez określony
czas pozwala na obserwację jego cech unikalnych i~wzorców zachowania podczas
stygnięcia.
Następnie należy wybrać czas nagrywania materiałów wideo kamerą.
Na podstawie wstępnych obserwacji i~próbnych nagrań zdecydowano się na
ogrzewanie próbek przez jedną minutę oraz rejestrację ich stygnięcia przez
cztery minuty.
Taka konfiguracja daje przy badanych pyłach rud miedzi ostry i~szczegółowy
obraz w~początkowej fazie nagrywania oraz widocznie rozmazany i~mniej
kontrastowy materiał pod koniec stygnięcia próbek.
Charakter procesu przejścia między tymi stanami pozwoli na klasyfikację
badanych próbek.
Ostatnia decyzja kształtująca charakter pomiarów dotyczy chwili
przechwytywania stopklatek z~pozyskanych materiałów wideo.
Na podstawie obserwacji zdecydowano się eksportować 5 klatek na początku
każdej minuty nagrania.
Po ustaleniu planu eksperymentu pomiarowego przystąpiono do jego wykonania.
Zgodnie z~opisem badanych materiałów w~sekcji \ref{sec:grains}, zgromadzono
materiały wideo dla czterech klas ziaren rudy miedzi.
Ze względu na czasochłonność pomiarów dla każdej klasy materiału nagrano trzy
materiały wideo.
Przy nagrywaniu stygnięcia tej samej klasy materiały pomiędzy pomiarami
próbkę poddawano przemieszaniu, aby uniknąć powtarzania struktur
ułożenia ziaren dla tej samej klasy rud miedzi.
Z~pozyskanych nagrań wyeksportowano stopklatki używając programu FLIR Tools,
pamiętając o~używaniu algorytmu automatycznego doboru zakresu temperatur
zgodnie z~opisem w~podsekcji \ref{subsec:camsoft}.

Proces pomiarowy w~protypowych warunkach był obarczony brakiem dużej precyzji
i~powtarzalności.
Obserwacja uzyskanych nagrań pokazała, że obrazy tej samej klasy ziaren
niekoniecznie osiągały tę samą temperaturę po ogrzewaniu przez ustalony czas.
Problemem okazało się również dostosowanie ostrości obrazu z~kamery.
Jak wytłumaczono w~podsekcji \ref{subsec:lens} wykorzystywany aparat cechuje
się bardzo małą głębią ostrości, co powoduje że drobne ruchy kamery
mogą spowodować utratę czytelności obrazu.
Z~kolei proces pomiarowy wymagał ciągłego przenoszenia próbki między
stanowiskiem do podgrzewania materiału oraz nagrywania filmów.
Należy mieć na uwadze że w~czasie układania próbek pod kamerą oraz poprawiania
ostrości postępowało stygnięcie ziaren, co sprzyjało brakowi powtarzalności
pomiarów.
Ze względu na przypadkowe utraty ostrości przy nagrywaniu oraz zbyt długi czas
przenoszenia i~przygotowania próbki do nagrywania, eksperyment wymagał czasem
powtórnego nagrania próbki.
Ocenę jakości procesu pomiarowego przestawiono na podstawie jego użyteczności
w~klasyfikacji ziaren, pod koniec pracy.
Wnioski na temat stosowności i~rozmiaru zbioru danych przedstawiono
w~rozdziale \ref{ch:conclusions}.

\section{Analiza zebranych obrazów termowizyjnych}
Zgodnie z~zamysłem pomiarów przedstawionym w~sekcji \ref{sec:meas} z~każdego
nagrania wybrano pięć stopklatek.
Podczas eksperymentu uzyskano łącznie dwanaście pomiarów, zawierających
sumarycznie 60 zdjęć.
W~ramach uczenia maszynowego jest to bardzo mały zbiór danych, biorąc jednak
pod uwagę wstępno-badawczy charakter pracy oraz czasochłonność procesu 
pomiarowego zdecydowano, że jest to rozmiar zadawalający do pierwszych
prób klasyfikacji.
% TODO: może przenieść do glosariusza
Zebrane materiały mają format JPEG, do oznaczania zdjęć przyjęto schemat
nazw jak w~przykładzie: 115\_E11R\_1, gdzie człony nazwy oznaczają kolejno:
\begin{itemize}
	\item automatyczny numer nagrania w~programie FLIR Tools,
	\item klasę próbki,
	\item minutę nagrania.
\end{itemize}

\subsection{Prezentacja przykładowej serii pomiarowej}
Jak opisano w rozdziale \ref{sec:meas} jedna próbka w~zbiorze danych 
składa się z~serii pięciu zdjęć o~malejącym kontraście i~szczegółowości.
Na rysunku \ref{fig:sample} przedstawiono przykładową próbkę 104 klasy E5R.
Widoczny jest proces stygnięcia materiału.
Skala po prawej stronie obrazów ma malejące na kolejnych obrazach wartości
co pokazuje że następuje zmniejszenie temperatury na całym obrazie.
Dodatkowo obraz staje się coraz mniej wyraźny i~kontrastowy.
Ze względu na budowę materiału nagrzana próbka emituje ciepłoze swoich
zróżnicowanych struktur w~niejednorodny sposób.
Wraz z~ochłodzeniem próbki jej temperatura się wyrównuje i~kamera termowizyjna
rejestruje coraz mniej szczegółów.
Detale i~elementy charakterystyczne obrazu zlewają się na kolejnych zdjęciach,
wraz z~opadaniem temperatury na zdjęciu pojawia się także coraz więcej szumów.
\begin{figure}[htbp]
	\centering
	\begin{subfigure}{0.45\textwidth}
		\centering
		\includegraphics[width=\textwidth]{sample/104_E5R_0}
		\caption{Obraz z~próbki 104\_E5R\_0}
	\end{subfigure}
	\hspace{0.75cm}
	\vspace{0.5cm}
	\begin{subfigure}{0.45\textwidth}
		\centering
		\includegraphics[width=\textwidth]{sample/104_E5R_1}
		\caption{Obraz z~próbki 104\_E5R\_1}
	\end{subfigure}
	\begin{subfigure}{0.45\textwidth}
		\centering
		\includegraphics[width=\textwidth]{sample/104_E5R_2}
		\caption{Obraz z~próbki 104\_E5R\_2}
	\end{subfigure}
	\hspace{0.75cm}
	\vspace{0.5cm}
	\begin{subfigure}{0.45\textwidth}
		\centering
		\includegraphics[width=\textwidth]{sample/104_E5R_3}
		\caption{Obraz z~próbki 104\_E5R\_3}
	\end{subfigure}
	\begin{subfigure}{0.45\textwidth}
		\centering
		\includegraphics[width=\textwidth]{sample/104_E5R_4}
		\caption{Obraz z~próbki 104\_E5R\_4}
	\end{subfigure}
	\caption{Zdjęcia procesu stygnięcia w~przykładowej próbce 104 klasy E5R}
	\label{fig:sample}
\end{figure}

\subsection{Przetwarzanie danych wizyjnych}

\subsection{Poprawa jakości obrazu}

\subsection{Automatyczny odczyt zakresu pomiarowego temperatur z~obrazu}
Jak wspomniano w~sekcji \ref{subsec:camsoft} jednym z~kluczowych czynników
decydujących o~wyglądzie obrazów pochodzących z~kamery jest zakres temperatur
mapowany na kolory w~obrazie.
Niestety aplikacja FLIR Tools nie pozwala na eksport zakresu temperatur wraz
ze zdjęciami w~formie liczbowej.
W~czasie zapisu zdjęć oprogramowanie dodaje na nich interfejs z~aktywną skalą
pomiarową, jednak jest on graficznie naniesiony na obraz.
Aby ułatwić w przyszłości pracę z~materiałami z~kamery opracowano dodatkowo
mechanizm ekstrakcji zakresu temperatur z~zdjęć pochodzących z kamery.

W~celu konstrukcji funkcji odczytu wartości liczbowych z~obrazu posłużono
się gotową siecią neuronową zaprojektowaną do detekcji tekstu na zdjęciach.
Zdecydowano się na użycie popularnej biblioteki \emph{Pytesseract}.
Aby poprawnie odczytać wartości z~obrazu najpierw przycięto je tak by w~kadrze
znajdowała się tylko odczytywana liczba.
Ponieważ przy eksporcie zdjęć program FLIR Tools nakłada interfejs na zdjęcia
w~identyczny sposób, kadrowanie obrazu jest takie same dla każdej próbki
pomiarowej.
Wycięte kadry są bardzo małej rozdzielczości, aby ułatwić sieci rozpoznawanie
liczb zdecydowano się przeskalować je w~górę.
W~czasie skalowania włączono mechanizm anty aliasingu aby wyrównać krawędzie
cyfr.
Ponieważ używana sieć uznaje za tło kolor biały oraz poszukuje liczb w~kolorze
czarnym barwy na zdjęciu odwrócono.
Następnie obraz poddano binaryzacji metodą \emph{otsu}.
Jest to popularna i~wydajna metoda binaryzacji, jej efektywność jest
maksymalna kiedy ilość pikseli tła oraz pierwszego planu jest zbliżona,
dlatego poprawne kadrowanie liczb sprzyja jakości ich binaryzacji\cite{sezgin}.
Na rysunku \ref{fig:temp_bounds} przedstawiono kolejne etapy przygotowania
obrazu do rozpoznania liczb.
Implementację opisanego mechanizmu odczytywania zakresu temperatur ze zdjęć
przedstawiono na listingu \ref{lst:temp_bounds}.
\begin{figure}[htbp]
	\centering
	\begin{subfigure}{0.45\textwidth}
		\centering
		\includegraphics[width=\textwidth]{temp_bounds_scale}
		\caption{Przeskalowany kadr z~liczbą}
		\label{fig:temp_bounds_scale}
	\end{subfigure}
	\hspace{0.5cm}
	\begin{subfigure}{0.45\textwidth}
		\centering
		\includegraphics[width=\textwidth]{temp_bounds_bin}
		\caption{Kadr z~liczbą po binaryzacji}
		\label{fig:temp_bounds_bin}
	\end{subfigure}
	\caption{Przygotowanie zakresu temperatur do odczytu przez sieć neuronową}
	\label{fig:temp_bounds}
\end{figure}

% TODO: This code might be not ready
\begin{listing}[htbp]
\begin{minted}{python}
def get_temperature_bounds(img, bounds=(((6, 24), (283, 318)),
                                       ((219, 236), (283, 318)))):
    '''Extract temperature values from FLIR UI on image.'''
    img = invert(img)
    temp_txt = []
    for bound in bounds:
        bound_img = img[slice(*bound[0]), slice(*bound[1])]
        bound_img = rescale(bound_img, 4, anti_aliasing=True)
        thr = threshold_otsu(bound_img)
        img_txt = bound_img > thr
        img_txt = Image.fromarray(img_txt)
        temp = pytesseract.image_to_string(img_txt)
        if temp is not '': 
            temp = float(temp)
        else:
            temp = 0
        temp_txt.append(temp)
    return temp_txt
\end{minted}
\caption{Funkcja języka Python do odczytywania zakresu temperatur ze zdjęć
         z~kamery}
\label{lst:temp_bounds}
\end{listing}

\section{Poszukiwanie zależności użytecznych w~klasyfikacji}
Aby móc klasyfikować dane należy zastanowić się nad cechami które je
odróżniają.
Na rysunku \ref{fig:samplecompare} przedstawiono porównanie stygnięcia
dwóch rodzajów próbek: E5R oraz E6R.
W~klasyfikacji użyteczne będą dane które są unikalne dla danej klasy.
Ponieważ dane stanowią serię obrazów postępującego studzenia materiału, aby
wykorzystać pełnię możliwości zebranych zdjęć warto szukać cech
charakterystycznych dla przebiegu procesu chłodzenia.
\begin{figure}[htbp]
	\centering
	\begin{subfigure}{0.3\textwidth}
		\centering
		\includegraphics[width=\textwidth]{sample/104_E5R_0}
		\caption{Próbka 104\_E5R\_0}
	\end{subfigure}
	\hspace{0.25cm}
	\vspace{0.5cm}
	\begin{subfigure}{0.3\textwidth}
		\centering
		\includegraphics[width=\textwidth]{sample/104_E5R_1}
		\caption{Próbka 104\_E5R\_1}
	\end{subfigure}
	\hspace{0.25cm}
	\begin{subfigure}{0.3\textwidth}
		\centering
		\includegraphics[width=\textwidth]{sample/104_E5R_2}
		\caption{Próbka 104\_E5R\_2}
	\end{subfigure}
	\begin{subfigure}{0.3\textwidth}
		\centering
		\includegraphics[width=\textwidth]{sample/104_E5R_3}
		\caption{Próbka 104\_E5R\_3}
	\end{subfigure}
	\hspace{0.25cm}
	\vspace{0.5cm}
	\begin{subfigure}{0.3\textwidth}
		\centering
		\includegraphics[width=\textwidth]{sample/104_E5R_4}
		\caption{Próbka 104\_E5R\_4}
	\end{subfigure}
	\hspace{0.25cm}
	\begin{subfigure}{0.3\textwidth}
		\centering
		\includegraphics[width=\textwidth]{sample/117_E6R_0}
		\caption{Próbka 117\_E6R\_0}
	\end{subfigure}
	\begin{subfigure}{0.3\textwidth}
		\centering
		\includegraphics[width=\textwidth]{sample/117_E6R_1}
		\caption{Próbka 117\_E6R\_1}
	\end{subfigure}
	\hspace{0.25cm}
	\vspace{0.5cm}
	\begin{subfigure}{0.3\textwidth}
		\centering
		\includegraphics[width=\textwidth]{sample/117_E6R_2}
		\caption{Próbka 117\_E6R\_2}
	\end{subfigure}
	\hspace{0.25cm}
	\begin{subfigure}{0.3\textwidth}
		\centering
		\includegraphics[width=\textwidth]{sample/117_E6R_3}
		\caption{Próbka 117\_E6R\_3}
	\end{subfigure}
	\begin{subfigure}{0.3\textwidth}
		\centering
		\includegraphics[width=\textwidth]{sample/117_E6R_4}
		\caption{Próbka 117\_E6R\_4}
	\end{subfigure}
	\caption{Porównanie procesu stygnięcia próbek klasy E5R oraz E6R}
	\label{fig:samplecompare}
\end{figure}

\subsection{Wybór cech obrazu użytecznych w~klasyfikacji}
\label{subsec:featureextr}
Po przyjrzeniu się rysunkowi \ref{fig:samplecompare} widoczne  jest, że
w~próbce E6R temperatura ziaren zaczęła wyrównywać się szybciej.
Na podstawie obserwacji zebranych danych rozpatrzono następujące możliwości
obserwacji cech charakterystycznych materiałów:
\begin{enumerate}[a)]
	\item \label{it:imgcnn} 
	      klasyfikacja zdjęć w~całości jako macierzy danych przez złożoną sieć
	      konwolucyjną,
	\item \label{it:fft} 
	      analiza częstotliwościowa obrazów w~celu śledzenia tempa rozmycia
	      kolejnych zdjęć,
	\item \label{it:glcm} 
	      użycie macierzy GLCM jako wejścia sieci neuronowych,
	\item \label{it:edge}
	      wykrywanie krawędzi ziaren i~wyznaczanie reprezentacji liczbowej
	      ich kształtów oraz powierzchni,
	\item \label{it:blob}
	      śledzenie zlewania się i~zanikania małych detali na obrazie.
\end{enumerate}

Wszystkie przedstawione opcje mają uzasadnienie i~mogą sprawdzić się dobrze
jako podstawa klasyfikacji.
Należy jednak ocenić której z~nich użyć w~pierwszej próbie konstrukcji
klasyfikatora. Metoda \ref{it:imgcnn}, z~użyciem sieci konwolucyjnych może
wykorzystywać najnowsze rozwiązania w~dziecinie uczenia maszynowego,
jednak przy jej użyciu na przeszkodzie może stać bardzo mały rozmiar zbioru
uczącego.
Na niewielu zgromadzonych obrazach znajduje się wiele informacji i~szumów,
a~złożoność jednego zdjęcia jest na tę chwilę nieproporcjonalna do wielkości
zbioru danych.
Pomysł ten można spróbować zrealizować po rozszerzeniu pomiarów.
Kolejna opcja \ref{it:fft} z~użyciem analizy częstotliwościowej wymaga
złożonych operacji matematycznych i~może być wrażliwa na szumu na obrazie.
Po analizie innych opcji zdecydowano, że istnieją bardziej obiecujące
alternatywy.
Macierz glcm (\textit{Gray-Level Co-Occurrence Matrix}), na której może
bazować opcja \ref{it:glcm}, to tablica zawierająca informacje o~relacjach
wszystkich par pikseli na obrazie.
Pozwala ona na analizę takich wartości jak: kontrast, korelacja, energia
oraz homogeniczność.
Jest to opcja dająca możliwość analizy dużej ilości informacji, z~pewnością
warta rozpatrzenia, jednak dosyć skomplikowana.
Na obrazie można także wykrywać kształty ziaren za pomocą filtrów detekcji
krawędzi.
Opcję tę testowano przy pomocy filtra \emph{Canny}.
Krawędzie ziaren okazały się jedank trudne do wykrycia kształtów i~dalszej
segmentacji ze względu na małą rodzielczość oraz duże upakowanie ziaren.
Operacje morfologiczne domykania kształtów powodowały bardzo duże zmiany
w~obrazie i~zlewały ziarna.
Rozwój takiego podejścia przy analizowanych obrazach wymaga zaawansowanej
i~ostrożnej obróbki zdjęć.
Ostatnia opcja \ref{it:blob} wynika z~obserwacji detali na obrazach.
Na przedstawionych zdjęciach próbek można zauważyć drobne ciemne punkty,
które są obszarami o~wolniejszej wymianie ciepła z~otoczeniem niż reszta
powierzchni ziaren.
Na rysunku \ref{fig:blobdetail} przedstawiono zbliżenie na grupę takich
detali, w~czterech częściach procesu stygnięcia.
\begin{figure}[htbp]
	\centering
	\begin{subfigure}{0.3\textwidth}
		\centering
		\includegraphics[width=\textwidth]{sample/detail_119_E5R_0}
		\caption{Grupa detali w~próbce 119\_E5R\_0}
	\end{subfigure}
	\hspace{0.25cm}
	\centering
	\begin{subfigure}{0.3\textwidth}
		\centering
		\includegraphics[width=\textwidth]{sample/detail_119_E5R_1}
		\caption{Grupa detali w~próbce 119\_E5R\_1}
	\end{subfigure}
	\hspace{0.24cm}
	\begin{subfigure}{0.3\textwidth}
		\centering
		\includegraphics[width=\textwidth]{sample/detail_119_E5R_2}
		\caption{Grupa detali w~próbce 119\_E5R\_2}
	\end{subfigure}
	\caption{Zbliżenie na charakterystyczne grupy detali materiału}
	\label{fig:blobdetail}
\end{figure}
Analizując próbki można zauważyć, że wraz ze stygnięciem ciemne punkty
w~grupach zlewają się, a~następnie zanikają.
Dodatkowo ich liczba na poszczególnych klasach materiałów jest różna.
Zdecydowano się na wybór metody polegającej na śledzeniu liczby tych punktów
i~ich zaniku.
Taka analiza wiąże się z~przetwarzaniem obrazów i~utworzeniem algorytmów
śledzenia detali.
Opcja ta wydaje się jednak obiecująca, ponieważ nawet podczas wstępnej
obserwacji próbek można dopatrywać się zależności miedzy klasami a~obecnością
omawianych detali.
Można również zaobserwować, że w~czasie stygnięcia na kolejnych zdjęciach
pojawiają się sporadycznie także nowe detale.
Zjawisko ich powstawania jest jednak pomijalne przy zdecydowanej tendencji
do zanika i~rozmywania plam, którą omówiono.
Fakt pojawiania się nowych detali wzięto pod uwagę przy późniejszym procesie
projektowania algorytmów ich śledzenia.

\subsection{Wybór algorytmu detekcji ziaren} \label{subsec:blobdetect}
Zgodnie z~rozważaniami przedstawionymi w~podsekcji \ref{subsec:featureextr}
w celu klasyfikacji ziaren zdecydowano się na obserwację ilości ciemnych,
drobnych detali na obrazach.
Należy więc wybrać metodę detekcji charakterystycznych punktów.
W~wykrywaniu omawianych detali użyteczne są algorytmy wykrywania plam na
podstawie analizy pochodnych wartości na obrazie.
Biblioteka Scikit-image udostępnia trzy algorytmy tego typu wykorzystujące:
\begin{enumerate}[a)]
	\item laplasjan funkcji Gaussa,
	\item różnicę funkcji Gaussa,
	\item wyznacznik Hesjanu.
\end{enumerate}
Algorytmy te pozwalają na wykrycie na obrazie plam o~kształcie zbliżonym
do kolistego, oraz oszacowanie ich promienia.
Stosuje się je na przykład w~analizie obrazów astronomicznych do detekcji
i~zliczania ciał niebieskich na zdjęciach wykonanych z~użyciem teleskopów.
% TODO: cite
Badane detale, przedstawione na rysunku \ref{ch:conclusions},
mają kształt zbliżony do kolistego, więc próba użycia rozpatrywanych
algorytmów jest uzasadniona.
Należy porównać dostępne warianty detekcji i~wybrać algorytm, który
działa najskuteczniej na zgromadzonych próbkach.

Metoda bazująca na laplasjanie funkcji Gaussa jest najdokładniejsza, ale
także najwolniejsza.
Funkcja Gaussa, której wykres ma charakterystyczny kształt krzywej dzwonowej
jest dana wzorem \ref{eq:gaussian}, gdzie:
\begin{itemize}
	\item $ \sigma $ to odchylenie standardowe,
	\item $ \mu $ to wartość średnia.
\end{itemize}
\begin{equation}
	P(x) = \frac{1}{{\sigma \sqrt {2\pi } }}
	e^{{{ - \left( {x - \mu } \right)^2 }
	\mathord{\left/ {\vphantom {{ - \left( {x - \mu } \right)^2 }
	{2\sigma ^2 }}} \right. \kern-\nulldelimiterspace} {2\sigma ^2 }}}
\label{eq:gaussian}
\end{equation}
Laplasjan to operator różniczkowy drugiego rzędu.
Omawiany algorytm oblicza wartości funkcji Gaussa dla coraz większego
odchylenia standardowego i~układa je w~sześcianie.
Poszukiwane plamy to lokalne maksima w~tym sześcianie.
Wadą tego rozwiązania jest bardzo wolne wykrywanie dużych plam z~powodu
złożoności obliczeniowej.

Różnica funkcji Gaussa jest metodą podobną do poprzedniej.
Ponownie rozmywa ona obrazu z~narastającymi odchyleniami standardowymi
z~użyciem funkcji Gaussa.
Następnie różnice rozmytych obrazów są układane w~sześcianie, gdzie
maksima to plamy.
Metoda ta jest szybsza i~mniej dokładna od algorytmu bazującego na laplasjanie
funkcji Gaussa, ale podobnie jak ona jest wolna w wykrywaniu dużych elementów.

Ostatnia metoda jest najszybsza, ale najmniej dokładna.
Polega ona na wyszukiwaniu maksimów w~macierzy Hesjanu, jest to macierz
drugich pochodnych cząstkowych.
Postać takiej macierzy w~n-wymiarowej przestrzeni zmiennych $ x $ przedstawia
wzór \ref{eq:hessian}.
Prędkość tej metody nie zależy od wielkości wykrywanych plam, ale małe
elementy mogą nie zostać przez nią wykryte.
\begin{equation}
	H = \begin{bmatrix}
	\dfrac{\partial^2 f}{\partial x_1^2} & 
	\dfrac{\partial^2 f}{\partial x_1\,\partial x_2} & 
	\cdots & \dfrac{\partial^2 f}{\partial x_1\,\partial x_n} \\[2.2ex]
	\dfrac{\partial^2 f}{\partial x_2\,\partial x_1} &
	\dfrac{\partial^2 f}{\partial x_2^2} &
	\cdots & \dfrac{\partial^2 f}{\partial x_2\,\partial x_n} \\[2.2ex]
	\vdots & \vdots & \ddots & \vdots \\[2.2ex]
	\dfrac{\partial^2 f}{\partial x_n\,\partial x_1} &
	\dfrac{\partial^2 f}{\partial x_n\,\partial x_2} &
	\cdots &
	\dfrac{\partial^2 f}{\partial x_n^2}
	\end{bmatrix}
\label{eq:hessian}
\end{equation}

Porównanie działania wymienionych metod przedstawiono na grafice
\ref{fig:blobcompare}.
Jak można było spodziewać się po opisie funkcji najlepsza okazała się metoda
Laplasjanu funkcji Gaussa.
Funkcję korzystającą z~wyznacznika Hesjanu należy odrzucić, ponieważ
w~rozważanym przypadku istotne jest wykrywanie małych plam.
Z~tego samego powodu korzystne jest użycie najbardziej dokładnej funkcji.
Ponieważ program nie powinien wykrywać dużych elementów nie ma ryzyka zbyt
powolnych obliczeń na obszernych plamach.
\begin{figure}[htbp]
	\centering
	\begin{subfigure}[t]{0.3\textwidth}
		\centering
		\includegraphics[width=\textwidth]{example-image}
		\caption{Laplasjan funkcji Gaussa}
	\end{subfigure}
	\hspace{0.25cm}
	\centering
	\begin{subfigure}[t]{0.3\textwidth}
		\centering
		\includegraphics[width=\textwidth]{example-image}
		\caption{Różnica funkcji Gaussa}
	\end{subfigure}
	\hspace{0.25cm}
	\begin{subfigure}[t]{0.3\textwidth}
		\centering
		\includegraphics[width=\textwidth]{example-image}
		\caption{Wyznacznik Hesjanu}
	\end{subfigure}
	\caption{Porównanie bibliotecznych algorytmów wykrywania plam w~obrazie}
	\label{fig:blobcompare}
\end{figure}

Aby wybrana funkcja laplasjanu funkcji Gaussa wykrywała, zgodnie
zamierzeniami tylko małe plamy należy podać jej odpowiednie parametry,
co uczyniono już na etapie porównania metod detekcji.
Wywołanie omawianej funkcji ma postać przedstawioną na listingu
\ref{lst:find_blobs}.
Funkcji podano dwa dodatkowe argumenty, które są istotne dla pożądanego
działania.
Argument \mintinline{python}{max_sigma=2} ogranicza odchylenie standardowe
obliczanych funkcji Gaussa, przez co wykrywane są tylko małe elementy.
Drugi argument \mintinline{python}{threshold=0.1} decyduje o~poziomie
powyżej jakiego punkt jest uznany za maksimum w~sześcianie laplasjanów.
Domyślna wartość tego argumentu \mintinline{python}{threshold=2.0} okazała 
się za duża, zmniejszono jej wartość aby wykrywać bardziej subtelne detale.
Na podstawie opisanego wywołania metody Laplasjanu funkcji Gaussa
opracowano funkcję zwracającą położenie i~promienie wykrytych plam.
\begin{listing}[htbp]
\begin{minted}{python}
def find_blobs(img):
    '''
    Find blobs in given image and get list of their positions and 
    radiuses.
    '''
    # Detect blobs with Difference of Gaussian
    blobs = blob_dog(img, max_sigma=2, threshold=0.1)
    # Get blobs radiuses from each kernel sigma
    blobs[:, 2] = blobs[:, 2] * sqrt(2)
    return blobs
\end{minted}
\caption{Funkcja języka Python do wykrywania detali w obrazie}
\label{lst:find_blobs}
\end{listing}

\subsection{Algorytm śledzenia ziaren w~serii zdjęć}
\label{subsec:blobtracking}
Użycie funkcji bibliotecznych opisanych w~podsekcji \ref{subsec:blobdetect}
pozwala na detekcję detali na pojedynczym zdjęciu.
Aby wykorzystać detekcję plam w~serii pięciu zdjęć które reprezentują
stygnięcie jednej próbki należy opracować zestaw funkcji śledzący zmiany 
na obrazie.
Na tym etapie prac niejednoznaczne jest jaki sposób obserwacji i~zliczania
detali będzie najlepszy do późniejszego prototypowania sieci neuronowej
klasyfikującej klasę ziaren.
Dlatego zdecydowano się na napisanie funkcji zliczania plam w~serii, tak by
możliwe były trzy warianty pracy algorytmu:
\begin{enumerate}[I.]
	\item \label{it:allblob}
	      zliczanie wszystkich detali na każdym etapie stygnięcia,
	\item \label{it:remainingblob}
	      zliczanie na kolejnych etapach stygnięcia jedynie tych detali,
	      które były obecne od początku stygnięcia,
	\item \label{it:percentblob}
	      zliczanie na kolejnych etapach pozostałego procentu ziaren, które
	      były obecne od początku stygnięcia.
\end{enumerate}

W~celu realizacji planu śledzenia detali utworzono funkcję języka Python.
Jej kod przedstawiono na listingu \ref{lst:find_blob_series}.
Aby umożliwić wariantowość algorytmu funkcja posiada opcjonalny parametr,
który pozwala włączyć lub wyłączyć śledzenie jedynie ziaren, które
były obecne od początku stygnięcia.
Przedstawiony kod wykorzystuje funkcję wykrywania detali w zdjęciu
przedstawioną na listingu \ref{lst:find_blobs}.
\begin{listing}[htbp]
\begin{minted}{python}
def find_blob_series(imgs, only_remaining=True):
    '''
    Return list of list of blobs found in each of given images.
    '''
    stages = []
    remaining = None
    for img in imgs:
        new_blobs = find_blobs(img)
        if remaining is not None and only_remaining:
            remaining = find_remaining_blobs(new_blobs, remaining)
        else:
            remaining = new_blobs
        stages.append(remaining)
    return stages
\end{minted}
\caption{Funkcja języka Python do śledzenia detali w~serii zdjęć}
\label{lst:find_blob_series}
\end{listing}
Jeżeli opcja detekcji jedynie ziaren obecnych od początku jest aktywna
zostaje wykorzystana utworzona funkcja porównania ziaren wykrytych
na obecnym etapie z~ziarnami obecnymi poprzednio.
Po analizie każdego zdjęcia nowy zestaw ziaren staje się zbiorem poprzednim
w~kolejnej iteracji.
Dzieje się tak, by uwzględnić fakt że podczas zlewania się ziaren środki
ciężkości i~promienie detali ulegają zmianie.
Dlatego kolejne grupy plam nie są zawsze porównywane z~grupą z~początku
serii, a~z~zestawem z poprzedniego analizowanego zdjęcia.

Na listingu \ref{lst:find_remaining_blobs} przedstawiono funkcję testującą
które z~detali znajdują się na kolejnych etapach stygnięcia.
\begin{listing}[htbp]
\begin{minted}{python}
def find_remaining_blobs(new_blobs, old_blobs):
    '''
    Return list of blobs present in both lists, where blob
    is considerd same if is in proximity of 2 times it's radius.
    '''
    remaining = []
    for new_blob in new_blobs:
        yn, xn, rn = new_blob
        for old_blob in old_blobs:
            yo, xo, ro = old_blob
            if inside_circle(xn, yn, xo, yo, 2 * ro):
                remaining.append(new_blob)
    return unique(remaining)
\end{minted}
\caption{Funkcja języka Python do wykrywania tych samych detali w~kolejnych
         obrazach}
\label{lst:find_remaining_blobs}
\end{listing}
Przyjmuje ona listy plam wykrytych w~dwóch kolejnych etapach iteracji.
Następnie następuje porównanie każdego ziarna z~nowej i~starej próbki.
Jeżeli nowy punkt znajduje się w~odległości do dwóch promieni od środka
starej plamy to zostaje on uznany za powtarzający się.
Powtarzające się punkty są dołączane do zwracanej listy.
Aby zrealizować ten zamysł zaimplementowano funkcję przedstawioną na listingu
\ref{lst:inside_circle}.
\begin{listing}[htbp]
\begin{minted}{python}
def inside_circle(x, y, a, b, r):
    '''
    Return True if point (x, y) lies inside circle 
    with center of (a, b) and radius r.
    '''
    return (x - a) * (x - a) + (y - b) * (y - b) < r * r
\end{minted}
\caption{Funkcja języka Python sprawdzająca czy dany punkt leży wewnątrz
         podanego okręgu}
\label{lst:inside_circle}
\end{listing}
Kod tej funkcji wynika wprost z~równania matematycznego okręgu, punkty
znajdują się w~jego wnętrzu jeśli spełniają nierówność przedstawioną we
wzorze, gdzie:
\begin{itemize}
	\item $ \left( x, y \right) $ to współrzędne danego punktu,
	\item $ \left( a, b \right) $ to współrzędne środka okręgu,
	\item $ r $ to promień okręgu.
\end{itemize}
\begin{equation}
\left( x - a \right) ^2 + \left( y - b \right )^2 < r^2
\label{eq:circle}	
\end{equation}
Ponieważ wiele punktów może znajdować się blisko siebie istnieje ryzyko,
że zostaną one zliczone wiele razy.
Z~tego powodu z~listy przed jej zwróceniem z~funkcji należy usunąć duplikaty.
W~tym celu utworzono funkcje pomocniczą widoczną na listingu \ref{lst:unique}.
\begin{listing}[htbp]
\begin{minted}{python}
def unique(multilist):
    '''Get list without repeating values.'''
    return list(set(tuple(i) for i in multilist))
\end{minted}
\caption{Funkcja języka Python usuwająca duplikaty z~listy}
\label{lst:unique}
\end{listing}
Najprostszym sposobem eliminacji powtarzających się wartości w~języki Python
jest konwersja listy na zbiór, który jest agregatem unikalnych 
nieuszeregowanych wartości.
Aby zwrócić z~funkcji listę należy z~powrotem skonwertować zbiór na typ
listowy.
W~naszym przypadku argument funkcji jest listą, której elementy są listami
zawierającymi współrzędne oraz promień każdej plamy.
Lista list jest nie podlega konwersji do zbioru ponieważ typ jej elementów
nie jest hashowalny.
Dlatego każdy punkt na liście zamieniono przed konwersją w~zbiór na typ
krotki.
W~ten sposób wszystkie konieczne przekształcenia są możliwe.
Aby zrealizować pomysł zliczania procentu plam pozostałych w~kolejnych etapach
stygnięcia zaimplementowano funkcje przedstawioną na listingu.
\ref{lst:percent}.
\begin{listing}[htbp]
\begin{minted}{python}
def percent_of_remaining_blobs_in_stages(stages):
    '''
    In each stage calculate the percentage of blobs that are
    present in the first stage and remain at given stage.
    '''
    num_of_blobs = [len(stage) for stage in stages]
    return [remaining / num_of_blobs[0] for remaining in num_of_blobs]
\end{minted}
\caption{Funkcja języka Python obliczająca ile procent ziaren z~początku
         stygnięcia pozostało w~jego kolejnych etapach}
\label{lst:percent}
\end{listing}

Przyglądając się zaimplementowanym funkcjom można docenić wybór języka
Python do realizacji projektu.
Elastyczność języka Python, w~szczególności jego list oraz wygoda
dynamicznego typowania pozwoliły na szybkie i~eleganckie zaimplementowanie
potrzebnych funkcji.
Wykorzystano bardzo zwięzły i~ekspresyjny element języka nazywany
\emph{wyrażeniami listowymi}.
Pozwala on na efektywne stosowanie jednolinijkowych pętli do przeglądania
i modyfikowania obiektów iterowalnych.
Użyteczność tego mechanizmu można zaobserwować w~przedstawionych listingach.
Podczas budowania funkcji dbano i~ich zwięzłość, dobry podział realizowanych
zadań oraz ograniczenie długości bloków kodu.
Dzięki temu program zachowuje zasady dobrego programowania.
Każdą funkcję opatrzono w~specjalny komentarz \emph{docstring} będący
standardem w~świecie języka Python.
Funkcje umieszczono w~odpowiednich plikach, które zawierają także
przykłady ich użycia i~obrazują działania oraz porównują efekty.
Kod przykładów posłużył do generowania wykresów i~obrazów zawartych
w~pracy.
% TODO: forma załączników
Pełen listing programów znajduje się w~załącznikach.

Przedstawiony zestaw funkcji daje możliwość ekstrakcji liczby plam
w~serii zdjęć na trzy sposoby.
Efekty działania utworzonego kodu przetestowano na obrazach pomiarowych.
Działanie sposobów zliczania ziaren \ref{it:allblob}~oraz
\ref{it:remainingblob} przedstawiono na rysunku \ref{fig:blobcount}.
Zaznaczono na nim wykryte detale, oraz wyróżniono wśród nich te które
są śledzone od początku etapu stygnięcia.
\begin{figure}[htbp]
	\centering
	\begin{subfigure}{0.45\textwidth}
		\centering
		\includegraphics[width=\textwidth]{example-image}
		\caption{Obraz z~próbki 104\_E5R\_0}
	\end{subfigure}
	\hspace{0.75cm}
	\vspace{0.5cm}
	\begin{subfigure}{0.45\textwidth}
		\centering
		\includegraphics[width=\textwidth]{example-image}
		\caption{Obraz z~próbki 104\_E5R\_1}
	\end{subfigure}
	\begin{subfigure}{0.45\textwidth}
		\centering
		\includegraphics[width=\textwidth]{example-image}
		\caption{Obraz z~próbki 104\_E5R\_2}
	\end{subfigure}
	\hspace{0.75cm}
	\vspace{0.5cm}
	\begin{subfigure}{0.45\textwidth}
		\centering
		\includegraphics[width=\textwidth]{example-image}
		\caption{Obraz z~próbki 104\_E5R\_3}
	\end{subfigure}
	\begin{subfigure}{0.45\textwidth}
		\centering
		\includegraphics[width=\textwidth]{example-image}
		\caption{Obraz z~próbki 104\_E5R\_4}
	\end{subfigure}
	\caption{Zliczanie śledzonych detali w~próbce dwoma sposobami}
	\label{fig:blobcount}
\end{figure}
Dodatkowo metodę śledzenia plam i~ich zanikania przedstawiono na rysunku
\ref{fig:blobremain}.
Pokazuje on położenie plam wykrytych w~kolejnych etapach stygnięcia.
Wykryte punkty zaznaczono na obrazie z~początku tego procesu.
Zmieniające się kolory obrazują detale wykryte w~kolejnych chwilach.
Pozawala to zaobserwować przemieszczanie się środków ciężkości i~promieni
plam podczas stygnięcia i zlewania się.
\begin{figure}[htbp]
    \centering
    \includegraphics[width=0.6\textwidth]{example-image}
    \caption{Wykryte detale, śledzone od początku stygnięcia, zaznaczone na
             pierwszym obrazie w~serii pomiarowej}
    \label{fig:blobremain}
\end{figure}
Działanie metody \ref{it:percentblob}, zliczającej procent detali, obecnych
od początku stygnięcia zobrazowano w~tabeli \ref{tab:blobpercent}.~
Działanie zliczania przedstawiono przedstawiono w~niej dla każdej 
rozpatrywanej klasy ziaren.
\begin{table}[htbp]
	\centering
	\begin{tabular}{c|c|c|c|c|c}
	\toprule
	Próbka & Minuta 0 & Minuta 1 & Minuta 2 & Minuta 3 & Minuta 4 \\ \midrule
	104\_E5R  & 1.0 & 0.52 & 0.29 & 0.23 & 0.20 \\
	106\_E11R & 1.0 & 0.12 & 0.02 & 0.00 & 0.00 \\
	107\_E6R  & 1.0 & 0.20 & 0.12 & 0.02 & 0.02 \\
	111\_E1XP & 1.0 & 0.05 & 0.01 & 0.01 & 0.01 \\
	\bottomrule
	\end{tabular}
\caption{Procent detali wykrytych w~kolejnych etapach stygnięcia, które
         były obecne na jego początku}
\label{tab:blobpercent}
\end{table}

\subsection{Wizualizacja zebranych cech i~ocena ich użyteczności}
\label{subsec:datavis}
W~podsekcji \ref{subsec:blobtracking} przedstawiono opracowane algorytmy
śledzenia ilości detali w~rozpatrywanych obrazach stygnięcia ziaren rud
miedzi.
Opracowane metody mają sens przy klasyfikacji, jeśli pozyskane cechy są
charakterystyczne dla rozpatrywanych klas.
Aby oszacować użyteczność pozyskanych cech należy przeprowadzić ich
wizualizację.
W~tym celu stworzono wykresy ilości wykrytych plam dla poszczególnych
metod ich zliczania.

Rysunek \ref{fig:blobchartall} przedstawia wykres procesu zanikania plam dla
wariantu z~wykrywaniem wszystkich detali na każdym etapie stygnięcia.
\begin{figure}[htbp]
    \centering
    \includegraphics[width=0.6\textwidth]{example-image}
    \caption{Ilość wykrytych detali na~każdym etapie stygnięcia, w~wariancie
             zliczającym wszystkie plamy}
    \label{fig:blobchartall}
\end{figure}
Wykreślone krzywe dla różnych klas materiałów przecinają się, w~wykreślonych
wartościach trudno na pierwszy rzut oka dojrzeć jakiekolwiek zależności
i~uporządkowanie.
Ilość charakterystycznych detali może być cechą rud miedzi, ze względu na
ich różny skład, jednak w~analizowanym przypadku nie jest to cecha dająca
nadzieję na dobre wyniki klasyfikacji.
Powodem takiej sytuacji jest duża losowość tego typu danych.
Niezależnie od własności materiału, jego ułożenie podczas pomiaru jest
przypadkowe co ma wpływ na bezwzględną liczbę zliczonych detali.
Na ilość widocznych plam wpływa również sposób ustawienia ostrości kamery.
Po przyjrzeniu się krzywym można jednak domniemywać, że pewną cechą
charakterystyczną jest nachylenie krzywych.
Wskazuje to, że bardziej widoczne mogą być względne cechy czasowe, a~nie
bezwzględna liczba zliczonych detali.

Drugi sposób zliczania ziaren przedstawiono na rysunku \ref{fig:blobchartrem}.
\begin{figure}[htbp]
    \centering
    \includegraphics[width=0.6\textwidth]{example-image}
    \caption{Ilość wykrytych detali na~każdym etapie stygnięcia, w~wariancie
             zliczającym wszystkie plamy}
    \label{fig:blobchartrem}
\end{figure}
W~tym przypadku zliczano jedynie detale obecne na obrazie od początku
stygnięcia.
Ten wariant śledzenia ilości plam daje dużo lepsze perspektywy na klasyfikację
ziaren.
Dla różnych typów próbek widoczne są charakterystyczne przebiegi krzywych.
Obserwacja dotycząca nachylenia krzywych staje się bardziej uzasadniona,
pochodne poszczególnych klas wydają się zbliżone.
Wyeliminowanie pojawiających się detali pomogła w~obserwacji procesu
stygnięcia i~zmniejszyła czynniki losowe.
Na podstawie analizy nagrań można domniemywać, że nowe ziarna pojawiały
się w wyniku zaobserwowanych delikatnych ruchów materiału.
Te mogły wynikać z~drgań stanowiska pomiarowego, istotnym czynnikiem może
być także mała głębia ostrości obiektywu.
W~późniejszych etapach stygnięcia na obrazach pojawia się także coraz więcej
szumów, które mogą zostać wykryte przez algorytm.
Śledzenie detali, które znajdują się na obrazie od początku eliminuje ten
problem.

Wyniki ostatniej metody polegającej na procentowym ustaleniu zaniku ilości
detali została przedstawiona na rysunku \ref{fig:blobcharperc}.
\begin{figure}[htbp]
    \centering
    \includegraphics[width=0.6\textwidth]{example-image}
    \caption{Ilość wykrytych detali, przy procentowym zliczaniu detali, które
             występowały w~serii od początku stygnięcia}
    \label{fig:blobcharperc}
\end{figure}
Widoczne jest odseparowanie różnych klas rud miedzi, różne typy próbek
charakteryzują się odmiennymi postępami rozmycia detali w~czasie.
Śledzenie ilości detali w~stosunku względnym dało najlepsze rezultaty
oraz pozwoliło na pewną normalizację danych, która była trudna do realizacji
przy bezwzględnym zliczaniu wszystkich plam na obrazie.
Należy także zauważyć, że przedstawione krzywe przedstawione na rysunkach
\ref{fig:blobchartrem} oraz \ref{fig:blobcharperc} mają kształt zbliżony
do krzywych eksponencjalnych.
Jest to obserwacja wskazująca, że pozyskane cechy dobrze oddają naturę procesu
stygnięcia.
Proces opadania temperatury ciał opisuje \emph{prawo stygnięcia Newtona},
które ma postać równania różniczkowego, przedstawionego we wzorze
% TODO: Cite
\ref{eq:newtonlawdiff}, gdzie:
\begin{itemize}
	\item $ T \left( t \right) $ to funkcja temperatury ciała w~czasie,
	\item $ T_R $ to temperatura otocznia,
	\item k to współczynnik liczbowy, charakterystyczny dla danego ciała.
\end{itemize}

\begin{equation}
	\frac{dT \left( t \right)}{dt}=-k\left( T \left( t \right) -T_{R} \right)
\label{eq:newtonlawdiff}
\end{equation}
Rozwiązanie równania przedstawia wzór \ref{eq:newtonlaw}, jak widać ma ono
charakter eksponencjalny, do którego zbliżone są wykresy na rysunkach
\ref{fig:blobchartrem} oraz \ref{fig:blobcharperc}.
\begin{equation}
	T(t) - T_{R} = \Delta T (t) = \Delta T (0) \ e^ {-k t}
\label{eq:newtonlaw}
\end{equation}

Analiza pozyskanych cech wskazuje że istnieje możliwość wykorzystania ich
do klasyfikacji rud miedzi.
Dalsza ocena wyników pracy będzie zależała od jakości działania sieci
neuronowej stworzonej do rozpoznawania przygotowanych danych.








